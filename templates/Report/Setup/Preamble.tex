%\usepackage{microtype}      % better looking text
%\usepackage[utf8]{inputenc}
%\usepackage[dvips]{graphicx}       % Required for including pictures
\usepackage{fontspec}       % Package for custom fonts
\usepackage[]{geometry}     % Package for changing page margins (before fancyhdr) 
\usepackage{fancyhdr}       % Package to change header and footer
\usepackage{parskip}        % Package to tweak paragraph skipping (instead of indents a small skip is added after every paragraph)
\usepackage{titlesec}
\usepackage{tikz}           % Package for drawing
\usetikzlibrary{calc, positioning, backgrounds, decorations.pathreplacing} % Required for drawing custom tikz shapes
\usepackage{pgfplots}       % Package for creating graphs and charts
\usepackage{xcolor}         % Package for defining DTU colours to be used
\usepackage{amsmath}        % For aligning equations among other
\usepackage{siunitx}        % SI units
\usepackage{listings}       % Package for inserting code, (before cleveref)
\PassOptionsToPackage{hyphens}{url} % Ability to line break urls at hyphens
\usepackage{hyperref}       % Package for cross referencing (also loads url package)
%\hypersetup{colorlinks=true, urlcolor=blue}   % overridden in Settings.tex
\usepackage{cleveref}       % improved cross referencing
\usepackage{textcomp}       % \textdegree = °C and other useful symbols
\usepackage[english]{babel} % localisation 
\usepackage{caption}        % better captions
\usepackage{subcaption}     % captions for subfigures
\usepackage{csquotes}       % For biblatex with babel
\usepackage[backend=biber,style=apa,sorting=nyt]{biblatex} % Package for bibliography (citing),                   
                            %                                citationstyle "apa": "author (year)"
                            %                                bibliography sorted by name, year, title (nyt)
\bibliography{bibliography.bib}   % Name of file that includes bibliography
\usepackage{tabularx}       % for ability to adjust column spacing in tabular better
\usepackage{colortbl}
\usepackage{booktabs}       % for better tables
\usepackage{threeparttable,threeparttablex}  % for tables with footnotes and more
\usepackage{caption}        % for customizing captions of figures and tables
\captionsetup[table]{skip=10pt}

\usepackage{float}          % floating figures in correct places
\usepackage{calc}           % Adds ability for latex to calculate (3pt+2pt) 
%\usepackage[printwatermark=false]{xwatermark} % Package for wartermark. Toggle printwatermark true or false to include or remove the watermark
\usepackage{blindtext}      % for filling a paragraph with rubbish text as a placeholder
\usepackage{wrapfig}        % for wrapping text around figures
\usepackage{rotating}       % for sidewaysfigures (rotated 90 degrees)

\usepackage{siunitx}        % for nice typesetting of numbers and ranges with units
\sisetup{separate-uncertainty, multi-part-units=single}     % multi-part-units may be repeat, single, or brackets
\usepackage[toc]{appendix}  % for better control over appendices, "toc" option to include heading "Appendices" in table of contents




% Additional packages

\usepackage[final]{pdfpages}   % To import pages from a pdf file. Use option "draft" to not import the pdf
\usepackage{doi}           % for better doi hyperlinks
\usepackage{longtable, ltcaption}  % for tables that span multiple pages
\usepackage{lipsum}        % for generating "lorem ipsum" text to be used as placeholder for unwritten sections
\usepackage{pstricks}      % for drawing diagrams and graphics programatically
\usepackage{rotfloat}      % for redefining new rotated floating environments

% New column types to control multiline text in tables
% See: https://tex.stackexchange.com/questions/12703/how-to-create-fixed-width-table-columns-with-text-raggedright-centered-raggedlef
\usepackage{array}
\newcolumntype{L}[1]{>{\raggedright\let\newline\\\arraybackslash\hspace{0pt}}m{#1}}
\newcolumntype{C}[1]{>{\centering\let\newline\\\arraybackslash\hspace{0pt}}m{#1}}
\newcolumntype{R}[1]{>{\raggedleft\let\newline\\\arraybackslash\hspace{0pt}}m{#1}}

\usepackage[version=4]{mhchem}  %  for typesetting chemical molecular formulae and equations
\usepackage{afterpage}    % for issuing commands after current page is typeset. e.g. use \afterpage{\clearpage} to flush out all floats before next text page is typeset.
\usepackage{placeins}     % defines a \FloatBarrier command, beyond which floats may not pass
\usepackage{textcomp}     % for additional symbols and glyphs
\usepackage{wasysym}      % for additional symbols and glyphs
\usepackage{outlines}     % for outline-style indented lists